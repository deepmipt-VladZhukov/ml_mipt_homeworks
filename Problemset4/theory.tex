\documentclass[11pt,a4paper]{article}
% \renewcommand\normalsize{\fontsize{12}{18.0pt}\selectfont}
\usepackage[utf8]{inputenc}
\usepackage[T2A]{fontenc}
\usepackage[russian]{babel}
\usepackage{graphicx}
\usepackage{amsmath,amssymb,amsthm}
\usepackage{epsfig}
\usepackage{import}
\usepackage{wrapfig}
\usepackage{epigraph}
\usepackage{verbatim}
\usepackage{soul}
\usepackage[usenames]{color}
\usepackage{listings}
\usepackage{pdfpages}

\usepackage[pdf]{graphviz}

%  ln code
\usepackage{amsmath}
\usepackage{mleftright}

\newcommand{\lnn}[1]{%
  \ln\left(#1\right)%
}

\newcommand{\lnb}[1]{%
  \ln\mleft(#1\mright)%
}
% code for ln

\pagestyle{empty}

\hoffset=-15mm    %по горизонтали влево на \hoffset
\voffset=-37.5mm  %по вертикали вверх на \voffset
\textheight=275mm
\textwidth=155mm


\newcommand{\eps}{\varepsilon}
\renewcommand{\phi}{\varphi}
\newcommand{\Impl}{\ensuremath{\Rightarrow}}
\newcommand{\RRR}{\overline{\mathbb{R}}}
\newcommand{\RR}{\mathbb{R}}
\newcommand{\NN}{\mathbb{N}}
\newcommand{\QQ}{\mathbb{Q}}
\newcommand{\nl}{\newline}

\usepackage{ifthen}
\newcommand\ifnonempty[2]{\ifthenelse{\equal{#1}{}}{}{#2}}
% Команда \task для условий задач с одним необязательным аргументом
% \defin определения, \exerc упражнения \prop предложения \theor теорема
\newcounter{task}
\newcounter{defin}
\newcounter{prop}
\newcounter{thm}
\newcounter{lem}
\newcommand{\task}[1][]{\smallskip\par\hangafter=1\normalsize\textbf{Задача \refstepcounter{task}\thetask\ifnonempty{#1}{ (#1)}.}~}
\newcommand{\defin}[1][]{\smallskip\par\hangafter=1\normalsize\textbf{Определение \refstepcounter{defin}\thedefin\ifnonempty{#1}{ (#1)}.}~}
\newcommand{\exerc}[1][]{\smallskip\par\hangafter=1\normalsize\textbf{Упражнение \refstepcounter{task}\thetask\ifnonempty{#1}{ (#1)}.}~}
\newcommand{\prop}[1][]{\smallskip\par\hangafter=1\normalsize\textbf{Предложение \refstepcounter{prop}\theprop\ifnonempty{#1}{ (#1)}.}~}
\newcommand{\thm}[1][]{\smallskip\par\hangafter=1\normalsize\textbf{Теорема \refstepcounter{thm}\thethm\ifnonempty{#1}{ (#1)}.}~}
\newcommand{\lem}[1][]{\smallskip\par\hangafter=1\normalsize\textbf{Лемма \refstepcounter{lem}\thelem\ifnonempty{#1}{ (#1)}.}~}
% \setcounter{thm}{2}

\begin{document}
\begin{center}
\Huge {
\noindent
\textbf{Теоретические задачи}
}
\end{center}

\Large {
\textbf {3.2 Вероятностный смысл регуляризаторов}
}
Рассмотрим вектор весов $$ w \in \mathbb{R}^n $$
Положим, чо веса распределены по нормальному закону со средним $0$ и дисперсией $\sigma^2$
 $$ f_w = \frac{1}{\sigma \sqrt{2 \pi}} exp (-\frac{||w||^2}{2\sigma}) \ \ \ (*) $$
Тогда максимизация правдоподобия $(*)$ будет эквивалентна минимизации
$$ -ln (f_w) = \frac{1}{2\sigma} ||w||_2 + ln(...) $$
Что в свою очередь эквивалентно минимищации
$$ ||w||_2 \to min $$
это и есть $L_2$ регуляризация по определению

Заметим, что распределение лапласа имеет такой же вид с точностью до констант, а единственным отличием является норма => проделав то же самое получаем
$$||w||_1 \to min $$
а это $L_1$ регуляризация по определению


\Large {
\textbf {3.3 SVM и максимизация разделяющей полосы}
}
Классификатор
$$ a(x) = sign(<w, x> - w_0) $$
Подгоним веса так чтобы:
$$ min_{i=1..l} y_i (<w, x_i> - w_0) = 1$$
В случае линейно разделимой выборки мы имеем некотрые точки $x_+, x_-$ которые лежат на границе полосы, тогда ширина полосы будет равна
$$ <x_+ - x_-, \frac{w}{||w||}> = $$
$$= \frac{<x_+, w> - <x_-, w>}{||w||} = \frac{2}{||w||}$$
В случае линейно неразделимой выборки получаем
$$
\begin{cases}
||w||_2 \to min\\
y_i (<w, x_i> - w_0) \ge 1, i=1..l
\end{cases}$$

$$
\begin{cases}
\frac{1}{2} ||w||_2 + C \sum_{i=1}^l\xi_i \to min_{w, w_0, \xi} \\
y_i (<w, x_i> - w_0) \ge 1 - \xi_i, i=1..l \\
\xi_i \ge 0, i = 1..l
\end{cases}
$$
Заметим, что
$$ y_i (<w, x_i> - w_0) = M_i $$
$$ \xi_i \ge 0$$
$$\xi \ge 1 - M_i $$

Тогда получаем, что $ \xi_i = (1 - M_i)_+ $

Следовательно безусловная задача оптимизации:
$$ Q(w, w_0) = \sum_{i=1}^l (1 - M_i(w, w_0))_+ + \frac{1}{2C} ||w||^2 \to min_{w, w_0} $$

\Large {
\textbf {3.4 Kenel trick}
}
$$ K(x, y) = x_1^2 y_1^2 + x_2^2 y_2^2 + 1 $$
$$ \phi(x) \to (x_1^2, x_2^2, 1) $$
$$ dim V = 3 $$

$$ (w_1, w_2, w_3) \phi(x_1, x_2, x_3) + w_0 =|_{(w_1, w_2, w_3) = (1, 2, 0), w_0 = -3}  = x_1^2 + 2x_2^2  - 3 = 0$$

\Large {
\textbf {3.6 Повторение метрики качества}
}
\\accacy - доля правильных ответов при классификации
\\precision - $\frac{tp}{tp + fp}$
\\recall - $\frac{tp}{tp + fn}$

\end{document}
