\documentclass[11pt,a4paper]{article}
% \renewcommand\normalsize{\fontsize{12}{18.0pt}\selectfont}
\usepackage[utf8]{inputenc}
\usepackage[T2A]{fontenc}
\usepackage[russian]{babel}
\usepackage{graphicx}
\usepackage{amsmath,amssymb,amsthm}
\usepackage{epsfig}
\usepackage{import}
\usepackage{wrapfig}
\usepackage{epigraph}
\usepackage{verbatim}
\usepackage{soul}
\usepackage[usenames]{color}
\usepackage{listings}
\usepackage{pdfpages}

\usepackage[pdf]{graphviz}


\pagestyle{empty}

\hoffset=-15mm    %по горизонтали влево на \hoffset
\voffset=-37.5mm  %по вертикали вверх на \voffset
\textheight=275mm
\textwidth=155mm


\newcommand{\eps}{\varepsilon}
\renewcommand{\phi}{\varphi}
\newcommand{\Impl}{\ensuremath{\Rightarrow}}
\newcommand{\RRR}{\overline{\mathbb{R}}}
\newcommand{\RR}{\mathbb{R}}
\newcommand{\NN}{\mathbb{N}}
\newcommand{\QQ}{\mathbb{Q}}
\newcommand{\nl}{\newline}

\usepackage{ifthen}
\newcommand\ifnonempty[2]{\ifthenelse{\equal{#1}{}}{}{#2}}
% Команда \task для условий задач с одним необязательным аргументом
% \defin определения, \exerc упражнения \prop предложения \theor теорема
\newcounter{task}
\newcounter{defin}
\newcounter{prop}
\newcounter{thm}
\newcounter{lem}
\newcommand{\task}[1][]{\smallskip\par\hangafter=1\normalsize\textbf{Задача \refstepcounter{task}\thetask\ifnonempty{#1}{ (#1)}.}~}
\newcommand{\defin}[1][]{\smallskip\par\hangafter=1\normalsize\textbf{Определение \refstepcounter{defin}\thedefin\ifnonempty{#1}{ (#1)}.}~}
\newcommand{\exerc}[1][]{\smallskip\par\hangafter=1\normalsize\textbf{Упражнение \refstepcounter{task}\thetask\ifnonempty{#1}{ (#1)}.}~}
\newcommand{\prop}[1][]{\smallskip\par\hangafter=1\normalsize\textbf{Предложение \refstepcounter{prop}\theprop\ifnonempty{#1}{ (#1)}.}~}
\newcommand{\thm}[1][]{\smallskip\par\hangafter=1\normalsize\textbf{Теорема \refstepcounter{thm}\thethm\ifnonempty{#1}{ (#1)}.}~}
\newcommand{\lem}[1][]{\smallskip\par\hangafter=1\normalsize\textbf{Лемма \refstepcounter{lem}\thelem\ifnonempty{#1}{ (#1)}.}~}
% \setcounter{thm}{2}

\begin{document}
\begin{center}
\Huge {
\noindent
\textbf{4 Теоретические задачи}
}
\end{center}
\Large {
\textbf {4.1 наивный байес и центроидный классификатор}
}

% $$ \underset{y}{asd} $$
$$ \underset{y}{argmax} (\prod_{k=1}^{n} \frac{1}{2\pi \sigma^2}e^{\frac{(x^{(k)} - \mu_{y_k})^2}{2 \sigma^2}}  Pr(y)) = $$
$$=\underset{y}{argmin} (\sum_{k=1}^{n} (x^{(k)} - \mu_{y_k})^2) $$
Посколько $Pr(y) = const$,
а максимум оставшейся функции является минимумом аргумента экспоненты, в которой также можно избавиться от константного множителя
\\
\Large{
\textbf {4.2 ROC-AUC случайных ответов}
}
Пусть количество 1 в выборке равно $k$, размер выборки $n$, тогда:
\\
$$ \xi_i = Bern(p) $$
$$ tp = \sum_{i=1}^{k} \xi_i,  fp = \sum_{i=1}^{n - k} \xi_i$$
$$ tn = \sum_{i=1}^{n - k} 1- \xi_i,  fn = \sum_{i=1}^{k} 1 - \xi_i$$
Получаем искомые случайные величины:
$$ FPR = \frac{fp}{fp + tn} = \frac{\sum_{i=1}^{n-k} \xi_i}{\sum_{i=1}^{n-k} \xi_i + \sum_{i=1}^{n-k} (1 - \xi_i)} = \frac{\sum_{i=1}^{n-k} \xi_i}{n-k}$$
$$ TPR = \frac{tp}{tp + fn} = \frac{\sum_{i=1}^{k} \xi_i}{\sum_{i=1}^k \xi_i + \sum_{i=1}^{k} (1 - \xi_i)} = \frac{\sum_{i=1}^{k}  \xi_i}{k}$$
взяв математическое ожидание, получим:
$$ \mathbb{E} FPR = p$$
$$ \mathbb{E} TPR = p$$

Значит средняя точка лежит на диагонали квадрата. Значит площадь в среднем равна $\frac{1}{2}$

\Large{
\textbf {4.3 Ошибка 1NN оптимального байесовского классификатора}
}
$$ \lim_{n \to \infty} Pr(y \ne y_n) = \lim_{n \to \infty} \sum_{y \ne y_n \in \{0, 1\}} Pr(y, y_n | x, x_n) =$$
$$= \lim_{n \to \infty} \sum_{y \ne y_n \in \{0, 1\}} Pr(y|x) Pr(y_n|x_n) = \sum_{y \in \{0, 1\}} Pr(y|x) (1 - Pr(y|x)) = $$
$$= Pr(0|x)(1 - Pr(0|x)) + Pr(1 | x) (1 - Pr(1 | x)) = $$
$$= 2 \max_{y \in \{0, 1\}} Pr(y|x) (1 - \max_{y \in \{0, 1\}} Pr(y|x)) \le 2E_B = $$
$$ 2 (1 - \max_{y \in \{0, 1\}} Pr(y|x))$$
\end{document}
