\documentclass[11pt,a4paper]{article}
% \renewcommand\normalsize{\fontsize{12}{18.0pt}\selectfont}
\usepackage[utf8]{inputenc}
\usepackage[T2A]{fontenc}
\usepackage[russian]{babel}
\usepackage{graphicx}
\usepackage{amsmath,amssymb,amsthm}
\usepackage{epsfig}
\usepackage{import}
\usepackage{wrapfig}
\usepackage{epigraph}
\usepackage{verbatim}
\usepackage{soul}
\usepackage[usenames]{color}
\usepackage{listings}
\usepackage{pdfpages}

\usepackage[pdf]{graphviz}

%  ln code
\usepackage{amsmath}
\usepackage{mleftright}

\newcommand{\lnn}[1]{%
  \ln\left(#1\right)%
}

\newcommand{\lnb}[1]{%
  \ln\mleft(#1\mright)%
}
% code for ln

\pagestyle{empty}

\hoffset=-15mm    %по горизонтали влево на \hoffset
\voffset=-37.5mm  %по вертикали вверх на \voffset
\textheight=275mm
\textwidth=155mm


\newcommand{\eps}{\varepsilon}
\renewcommand{\phi}{\varphi}
\newcommand{\Impl}{\ensuremath{\Rightarrow}}
\newcommand{\RRR}{\overline{\mathbb{R}}}
\newcommand{\RR}{\mathbb{R}}
\newcommand{\NN}{\mathbb{N}}
\newcommand{\QQ}{\mathbb{Q}}
\newcommand{\nl}{\newline}

\usepackage{ifthen}
\newcommand\ifnonempty[2]{\ifthenelse{\equal{#1}{}}{}{#2}}
% Команда \task для условий задач с одним необязательным аргументом
% \defin определения, \exerc упражнения \prop предложения \theor теорема
\newcounter{task}
\newcounter{defin}
\newcounter{prop}
\newcounter{thm}
\newcounter{lem}
\newcommand{\task}[1][]{\smallskip\par\hangafter=1\normalsize\textbf{Задача \refstepcounter{task}\thetask\ifnonempty{#1}{ (#1)}.}~}
\newcommand{\defin}[1][]{\smallskip\par\hangafter=1\normalsize\textbf{Определение \refstepcounter{defin}\thedefin\ifnonempty{#1}{ (#1)}.}~}
\newcommand{\exerc}[1][]{\smallskip\par\hangafter=1\normalsize\textbf{Упражнение \refstepcounter{task}\thetask\ifnonempty{#1}{ (#1)}.}~}
\newcommand{\prop}[1][]{\smallskip\par\hangafter=1\normalsize\textbf{Предложение \refstepcounter{prop}\theprop\ifnonempty{#1}{ (#1)}.}~}
\newcommand{\thm}[1][]{\smallskip\par\hangafter=1\normalsize\textbf{Теорема \refstepcounter{thm}\thethm\ifnonempty{#1}{ (#1)}.}~}
\newcommand{\lem}[1][]{\smallskip\par\hangafter=1\normalsize\textbf{Лемма \refstepcounter{lem}\thelem\ifnonempty{#1}{ (#1)}.}~}
% \setcounter{thm}{2}

\begin{document}
\begin{center}
\Huge {
\noindent
\textbf{Теоретические задачи}
}
\end{center}

\Large {
\textbf {2.1 Bias Varience decomposition}
}

$$\mathbb E_{x, y} \mathbb E_{X^l} (y - a_{X^l}(x))^2 = $$
$$ = \mathbb E_{x, y, X^l}(y - a_{X^l}(x))^2 =$$
$$ = \mathbb E_{x, y, X^l} (\mathbb E_{x, y, X^l} (y - a_{X^l})^2 | x) $$
$$ =\mathbb E_{x, y, X^l} ((y - a_{X^l})|x) = $$
$$ = \mathbb E_{x, y, X^l}((a_{X^l}(x) - f(x) - \eps)^2|x) =$$
$$ = \mathbb E_{x, y, X^l} ((a_{X^l(x)} - f(x))^2|x) = $$
$$ = \mathbb E_{x, y, X^l} ((a_X^l(x) - f(x))^2 |x) + 2 \mathbb E_{x, y, X^l}\eps \mathbb E_{x, y, X^l}(a_{X^l} - f(x)|x) + \mathbb E_{x, y, X^l}(\eps^2|x) = $$
Все без последнего слагаемого:
\\
$$ \mathbb E_{x, y, X^l} (a_{X^l} - \mathbb E_{x, y, X^l}a_{X^l}(x) + \mathbb E_{x, y, X^l} a_{X^l}(x) - f(x))^2 = $$
$$ \mathbb E_{x, y, X^l} [(a_{X^l(x) - \mathbb E_{x, y, X^l}a_{X^l}(x)})^2|x] + \mathbb E_{x, y, X^l}[\mathbb (E_{x, y, X^l} a_{X^l}(x) - f(x))^2|x]+$$
$$+ 2\mathbb E_{x, y, X^l} [(a_{X^l} - \mathbb E_{x, y, X^l}a_{X^l}) (\mathbb E_{x, y, X^l} a_{X^l} - f(x))|x]=$$
$$ \mathbb E_{x, y, X^l} [(a_{X^l(x) - \mathbb E_{x, y, X^l}a_{X^l}(x)})^2|x] + \mathbb E_{x, y, X^l}[\mathbb (E_{x, y, X^l} a_{X^l}(x) - f(x))^2|x]$$
То гда все выражение равно:
\\
$$\mathbb E_{x, y} (y - \mathbb E(y|x))^2 + \mathbb E_{x, y} (\mathbb E(y|x) - \mathbb E_{X^l}a_{X^l}(x))^2 + \mathbb E{x, y}\mathbb E{X^l}(a_{X^l}(X) - \mathbb E_{X^l}a_{X^l}(x))$$

\Large {
\textbf {2.3 Корреляция ответов базовых алгоритмов}
}
$$ \mathbb D_{X, Y, x, y} (\frac{1}{M} \sum_i \xi_i)= \frac{1}{M^2} \mathbb D_{X, Y, x, y} (\sum_i \xi_i)$$ по формеле дисперсии суммы
$$ = \frac{1}{M^2} M \mathbb D_{X, Y, x, y} \xi_1 + \frac{1}{M^2} \sum_{i \ne j} cov(\xi_i, \xi_j)=$$
$$ = \frac{1}{M} \sigma^2 + \frac{1}{M^2} \sum_{i \ne j} cov(\xi_i, \xi_j) =$$
$$ = \frac{1}{M} \sigma^2 + \frac{M(M - 1)}{M^2} \rho \sigma^2 = $$
$$ = \frac{1}{M} \sigma^2 + \frac{M - 1}{M} \rho \sigma^2 = $$
$$ = \rho \sigma^2 + \frac{\sigma^2}{M}(1 - \rho) $$
Что и требовалось
\end{document}
